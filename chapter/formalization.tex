\chapter{Formalization}\label{ch/formalization}

This section describes the development of a formalization of Homotopical Patch
Theory~\cite{Angiuli2016} in Cubical Agda. The development follows Angiuli et
al. in defining three patch theories of increasing complexity.

``An elementary patch theory'' uses the fundamental group of $S^1$ to implement
a patch theory of integers and the successor function, ``A patch theory with
laws'' extends to fixed-size vectors of strings and includes a patch law which
is used to implement a patch optimizer, and ``a patch theory with richer
contexts'' allows for more complex patch contexts with a repository indexed by
its history.

The aim of this chapter is to explain implementation choices alongside the key
code and definitions. The full implementation is available on
github~\footnote{\url{https://github.com/Aqissiaq/hpt-experiments}}.

Lastly, we look at some concrete computations using the formalization and
show that the elementary patch theory performs as expected, but that ``a
patch theory with laws'' and ``a patch theory with richer contexts''
require further development of Cubical Agda to fully explore.