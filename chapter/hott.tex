\chapter{Homotopy Type Theory}

In this chapter we introduce the basics of Homotopy Type Theory (HoTT). The
purpose is to give the reader the prerequisites to follow the formalization in
\autoref{ch/formalization}. For an excellent in-depth introduction
see Egbert Rijke's 2019 summer school notes~\cite{Rijke2019}. The canonical text
book is The Book~\cite{hottbook}.

We start by giving an intuitive introduction to dependent types and their
notation (in terms of inference rules). Then we consider two important
interpretations: types as propositions and types as spaces. We then move on to
inductive types and higher inductive types (HITS), before an introduction to the
syntax of the dependently typed language and proof assistant Agda.

Finally we look at a cubical interpretation of HoTT and Cubical Agda, an
implementation of it in Agda.