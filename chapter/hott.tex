\chapter{Homotopy Type Theory}
It's cool.~\cite{hottbook}

The purpose of this section is to give the enough prerequisites to follow the
ensuing development [pretentious af]. It is not a complete introduction to
Homotopy Type Theory. For a good introduction see Egbert Rijke's master thesis~\cite{Rijke2012}
and lecture notes~\cite{Rijke2019}, for a complete textbook see The
Book~\cite{hottbook}.

\subsection{Types and Propositions (and spaces?)}
\begin{enumerate}
  \item types represent propositions (and spaces)
  \item implication and simple and/or ($\rightarrow, \times, +$)
  \item quantifiers and dependent types (fibers) ($\Sigma, \Pi$)
\end{enumerate}

\subsection{Programs and Proofs (and terms?)}
\begin{enumerate}
  \item if types are propositions, how do we prove them?
  \item \emph{terms} of a type are \emph{proofs} of a proposition
\end{enumerate}

\subsection{Identity Types}
\begin{enumerate}
  \item what about things that are equal?
  \item J-rule (intuition: reflexive closure? groupoid structure?)
  \item paths in space
\end{enumerate}

\subsection{Higher Inductive Types}
\begin{enumerate}
  \item inductive types: base case(s) and point generator(s)
  \item example: $A + B, \mathbb{N}$
  \item HIGHER inductive types: terms and identities
  \item ie. points and paths between points (and paths between paths (and
    paths between paths between paths))
  \item elimination rules? they need to go somewhere, but this might not be it
\end{enumerate}

\subsection{Cubical?}
Why not take ``= is a path'' seriously?