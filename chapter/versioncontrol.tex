\chapter{Version Control Systems}

% They're not always cool.~\cite{git-inconsistent, badmerge}
A version control system (VCS) is a software development tool used to keep track
of data and changes to it -- often the data is the code for a software project.
With large projects, distributed teams and complex
dependencies between version this can be a challenge and a plethora of different
tools with a variety of theoretical approaches exist.

By far the most widely used tool is the distributed git~\cite{Git} which employs a
\emph{snapshot}-model. In this approach the system stores the state of the
repository, and computes patches in terms of line-by-line differences between
snapshots.

The distributed Mercurial~\cite{Mercurial} and centralized
Subversion~\cite{Subversion} work by similar models. These tools are fast, but
often poorly understood and have some un-intuitive
behavior~\cite{git-inconsistent, badmerge}.

An alternative to snapshot-models is a \emph{patch}-model. VCS with this model
instead store the set of patches explicitly, and then computes the repository
state from them. The most prominent example of such a system is
Darcs~\cite{Darcs} with its ``algebra of patches'', but we also mention
Pijul~\cite{Pijul} which takes a more categorical approach.

In this chapter we introduce the basic concepts and terminology, survey the
patch theory of Darcs as presented by Lynagh~\cite{Lynagh2006} and finally
explore ``homotopical patch theory''~\cite{Angiuli2016} -- an approach to VCS
using homotopy type theory and univalence.