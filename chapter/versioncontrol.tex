\chapter{Version Control Systems}

% They're not always cool.~\cite{git-inconsistent, badmerge}
A version control system (VCS) is a software development tool commonly used to keep track of
changes to a code base. With large projects, distributed teams and complex
dependencies between version this can be a challenge and a plethora of different
tools with a variety of theoretical approaches exist.

[NOT HAPPY WITH THIS]
By far the most common tool is the distributed git~\cite{Git} which employs
\emph{snapshot}-model in which the repository state is saved and changes are
recorded by storing the line-by-line difference with previous snapshots.
The distributed Mercurial~\cite{Mercurial} and centralized
Subversion~\cite{Subversion} work by similar models. These tools are fast, but
often poorly understood and have some un-intuitive
behavior~\cite{git-inconsistent, badmerge}.

An alternative to snapshot-models is a \emph{patch}-model in which the VCS
instead stores the changes and computes the repository state from them. The most
prominent example of such a system is Darcs~\cite{Darcs} with its ``algebra of
patches'', but we also mention Pijul~\cite{Pijul} which takes a more categorical approach.

In this chapter we introduce the basic concepts and terminology, investigate some
theoretical approaches to version control -- including Homotopical Patch Theory,
a formalization of which is covered in \autoref{ch/formalization} -- and mention
a novel approach based on type-theoretic co-equalizers which the author has not
succeeded in implementing.