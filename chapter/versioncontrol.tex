\chapter{Version Control Systems}

[PLAN. REMOVE THIS]
\begin{enumerate}
\item what even is a vcs?
\item approaches to vcs
  \begin{enumerate}
  \item git?
  \item darcs
  \item pijul/categorical
  \item hpt
  \end{enumerate}
\item a HoTT alternative
  \begin{enumerate}
  \item idea
  \item implementatoin
  \item ``result''
  \end{enumerate}
\end{enumerate}
% They're not always cool.~\cite{git-inconsistent}

% stolen from my own project proposal
Version control systems are ubiquitous in software development, where they
help facilitate cooperation and documentation of
the development process. Their basic use is to record (\emph{commit}) changes
to a codebase (\emph{repository}). Systems may also include ways for the
codebase to diverge (\emph{branch}) into different versions, and ways to
reunite (\emph{merge}) these versions.

In this chapter we introduce the basic concepts and terminology, investigate some
theoretical approaches to version control -- including Homotopical Patch Theory,
a formalization of which is covered in \autoref{ch/formalization} -- and mention
a novel approach based on type-theoretic co-equalizers which the author has not
succeeded in implementing.
