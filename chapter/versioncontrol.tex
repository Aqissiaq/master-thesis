\chapter{Version Control Systems}
They're not always cool.~\cite{git-inconsistent}

% stolen from my own project proposal
Version control systems are ubiquitous in software development, where they
help facilitate cooperation and documentation of
the development process. Their basic use is to record (\emph{commit}) changes
to a codebase (\emph{repository}). Systems may also include ways for the
codebase to diverge (\emph{branch}) into different versions, and ways to
reunite (\emph{merge}) these versions.

The purpose of this section is to introduce the terminology, requirements and
hopes for models of version control systems,

What do we need?
\begin{enumerate}
\item terms
  \begin{itemize}
  \item repository
  \item patch
  \item merge
  \item branch?
  \end{itemize}
\item requirements
  \begin{itemize}
  \item repo - accurately represents contents
  \item patch - applicable in a context, groupoid structure
  \item merge - ``pushout property''/reconcile, symmetric (for distributed
    systems), (do we need associativity as well?)
  \end{itemize}
\item hopes/goals
  \begin{itemize}
  \item repos - modular/composable, somehow polymorphic
  \item patches - \emph{semantic} in some sense
  \item merge - easily definable [sic.], considers semantics of patches
  \end{itemize}
\end{enumerate}