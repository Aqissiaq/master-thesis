\chapter{Introduction}

[SOME GENERAL INTRO HERE?]

\ChRef{ch/hott} contains an introduction to the Homotopy Type Theory (HoTT)
setting, a cubical interpretation of this type theory, and an introduction to
the syntax and workings of the Agda programming language~\cite{Agda} in general and Cubical
Agda~\cite{vezzosi2021cubical} in particular.

\ChRef{ch/vc} gives and exposition of Version Control Systems (VCS) and some
approaches to a theory of such systems. We give an account of
Darcs'~\cite{Darcs} ``Algebra of Patches'' as described by
Lynagh~\cite{Lynagh2006}, an approach based on category theory by Mimram and Di
Giusto~\cite{Categorical2013}, and Angiuli et al.'s ``Homotopical Patch
Theory''~\cite{Angiuli2016} which utilizes HoTT. Finally we present another
HoTT approach attempted by the author, which has not proven fruitful.

\ChRef{ch/formalization} describes the main contribution of this thesis: an
implementation of Homotopical Patch Theory in Cubical Agda. Following Angiuli et
al. we implement three patch theories of increasing complexity. In
\autoref{sec:results} we examine the implementations by testing them on simple
examples. We conclude that the theories and models behave as expected, but are
severely limited by the current limitations of Cubical Agda -- in particular
that \texttt{transp} and \texttt{hcomp} do not compute over indexed families.