\chapter{Introduction and Background}\label{ch/intro}

This thesis is about Homotopy Type Theory (HoTT), its interpretation and use in
Cubical Agda, and an application to version control systems (VCSs).
It is accordingly organized into three parts.

The remainder of this chapter contains an introduction to the Homotopy Type Theory
setting, a cubical interpretation of this type theory, and an introduction to
the syntax and workings of the Agda programming language~\cite{Agda} in general and Cubical
Agda~\cite{vezzosi2021cubical} in particular.

\ChRef{ch/vc} gives and exposition of Version Control Systems (VCS) and some
approaches to a theory of such systems. We give an account of
Darcs'~\cite{Darcs} ``Algebra of Patches'' as described by
Lynagh~\cite{Lynagh2006} and Angiuli et al.'s ``Homotopical Patch
Theory''~\cite{Angiuli2016} which utilizes HoTT.

\ChRef{ch/formalization} describes the main contribution of this thesis: an
implementation of Homotopical Patch Theory in Cubical Agda. Following Angiuli et
al. we implement three patch theories of increasing complexity. In
\autoref{sec:results} we examine the implementations by testing them on simple
examples. We conclude that the theories and models behave as expected, but are
restricted by the current limitations of Cubical Agda -- in particular that
\texttt{transp} and \texttt{hcomp} do not compute over inductive families.

Finally, \autoref{ch/conclusion} discusses the results of the formalization and
possible directions for future work on homotopical path theory in particular and
HoTT models of version control in general.

\section{Homotopy Type Theory}

The purpose of this introduction is to give the reader the prerequisites to
follow the formalization in \autoref{ch/formalization}. For an excellent
in-depth introduction see Egbert Rijke's 2019 summer school
notes~\cite{Rijke2019}. The canonical text is The Book~\cite{hottbook}.

We start by giving an intuitive introduction to dependent types and their
notation in terms of inference rules. Then we consider two important
interpretations: types as propositions and types as spaces. We then move on to
inductive types and higher inductive types (HITs), before an introduction to the
syntax of the dependently typed language and proof assistant Agda.

Finally we look at a cubical interpretation of HoTT and Cubical Agda, an
implementation of it in Agda.