\section{Spaces as Types}\label{sec:spaces}

Closely related to their groupoid structure another way to make sense of
identity types is through homotopy theory. With
this interpretation, pioneered by Voevodsky~\cite{voevodsky2014} and the HoTT
program~\cite{hottbook}, a type $A$ is a space in the sense of homotopy
types\footnote{\emph{Homotopy types} are a concept from homotopy theory, and only
related to the types of type theory in the sense described here.} and
a term of $x =_A y$ is a path in $A$ from $x$ to $y$.

In fact the collection of all such paths is itself a space (and thus a type):
the path space. Additionally there may be paths between paths, paths between
paths between paths and so on. These higher paths are the eponymous
``homotopies'' and provide a rich field of study on their own.
Geometrically we visualize them as ``filling in'' the space between paths.

Here, we present an overview of Voevodsky's model of types as spaces and his
influential concept of \emph{univalence}. A full exposition by
Kapulkin and Lumsdaine is found in~\cite{kapulkin2012}.

\subsection{Voevodsky's Model}
We have already seen a model that lets us view types as groupoids. When also
considering higher paths, types have the structure of ``weak higher-dimensional''
groupoids. A classical example of weak higher-dimensional groupoids can be found
in homotopy theory, where Kan complexes are a specific kind of ``well
behaved'' space. This was observed by both Voevodsky and Streicher around
2006~\cite{streicher2014sset} and Voevodsky used the insight to construct a
model for Martin-L\"of type theory.

Briefly, the model takes a combinatorial view of spaces by mapping them to
\emph{simplicial sets} -- a sort of arbitrary-dimensional triangulation. To
accommodate dependent types, we only consider Kan complexes -- intuitively the
simplicial sets where every triangle can be filled in.

The space containing all the basic spaces also forms a Kan complex
called a \emph{Universe} and often denoted by a $\mathcal{U}$.
Importantly the universe does not contain itself, but
may be contained in another universe ``one level up''. If necessary, a
cumulative hierarchy of universes containing all the lower ones is possible, but
Voevodsky settles for two.

\subsection{Univalence and Canonicity}%and the loss of canonicity
Voevodsky also shows that his model satisfies the \emph{univalence property}.
Loosely speaking this property says that the identity type for the universe is
equivalent to the type of equivalences. In other words, equivalent types can be
identified.

This property aligns well with mathematical practice where structures are often
considered up to some suitable notion of equivalence. With well defined types,
univalence means groups can be identified when they are isomorphic, categories
can be identified when they are equivalent and so on.
Consequently HoTT provides a setting for mathematics with desirable properties; the ``Univalent
Foundations'' program as introduced in The Book~\cite{hottbook} aims to do
mathematics in this setting.

Since the univalence property holds in Voevodsky's model, we might justifiably want to
include it in our type theory and The Book does exactly that by including the following
axiom (where idToEquiv is a function of type $(A =_{\mathcal{U}} B) \ra (A
\simeq B)$):
\begin{equation*}
  \begin{array}[t]{c}
    \Gamma \vdash A : \mathcal{U} \qquad \Gamma \vdash B : \mathcal{U}\\
    \hline
    \Gamma \vdash \operatorname{univalence}~A~B : \operatorname{isEquiv} (\operatorname{idToEquiv}~A~B)
  \end{array}
\end{equation*}
The inverse of idToEquiv is ua, with the computation rule that transport along
ua($f$) is propositionally equal to applying $f$.

However this poses a problem: when extending the
type theory with an axiom, the canonicity property is lost -- terms no longer
necessarily reduce to a normal form and we lose the nice constructive properties
of the resulting logic. The constructive/computational meaning of the univalence
axiom in homotopy type theory is an open area of research, and we will return to
it in \autoref{sec:cubical}.