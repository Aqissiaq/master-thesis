\section{Higher Inductive Types}\label{sec:HITs}

Now that we have several useful interpretations of types, we need some ways to
construct more complex types to study. In this section we look at one such way:
Higher Inductive Types (HITs).

\subsection{Inductive Types}
One way to construct more elaborate types is by induction. An inductive type is
defined by a number of constructors, which can be either constant terms or
functions. Let us return to the type of lists. It can be constructed from the
empty list and the function \texttt{cons} which takes an element and affixes it
to the start of a list. Using $[~]$ for the empty list and $::$ for the (infix)
cons function we have a pair of introduction rules:

\[
  \begin{array}[t]{c}
    \Gamma \vdash A \; \Type\\
    \hline
    \Gamma \vdash [~] : [A]\\
  \end{array}
  \qquad
  \begin{array}[t]{c}
    \Gamma \vdash a:A \qquad \Gamma \vdash as:[A] \\
    \hline
    \Gamma \vdash ( a :: as ) : [A]
  \end{array}
\]

From these we can construct arbitrarily long lists by starting with the empty
list and affixing new terms of $A$ to obtain $[~]$, $(a::[~])$, $(a'::(a::[~]))$
etc. For this reason we may refer to the introduction rules as ``constructors''
of $[A]$.

In order to use this type, we also need an induction principle. The
induction principle tells us how to use terms of the type by defining functions
into a family over it. If the family is constant (i.e. the function is
non-dependent) we call such a rule a ``recursion principle''.

\[
  \begin{array}{l}
    \Gamma, as : [A] \vdash P~as \; \Type\\
    \Gamma \vdash p_{[~]} : P~[~]\\
    \Gamma \vdash p_{::}: \Pi_{a:A} \Pi_{as:[A]} P~(a :: as)\\
    \hline
    \Gamma \vdash ind_{[A]}~p_{[~]}~p_{::} : \Pi_{[A]} P
  \end{array}
\]

This rule states that a dependent function out of $[A]$ can be constructed by
knowing what it does on the empty list, and -- given an $a:A$ and a list $as:[A]$ -- what
it does on $(a :: as)$. It is not a coincidence that we require two terms --
they correspond precisely to the two constructors.

The resulting function maps the empty list to $p_{[~]}$ and any arbitrary list
$(a::as)$ to $p_{cons}$ by the computation rules:
\[
  \begin{array}{c}
    \hline
    \Gamma \vdash ind_{[A]}~p_{[~]}~p_{::}~[~] \equiv p_{[~]}
  \end{array}
  \\
  \begin{array}{c}
    \hline
    \Gamma, a:A, as:[A] \vdash ind_{[A]}~p_{[~]}~p_{::}~(a::as) \equiv p_{::}~a~as
  \end{array}
\]

\subsection{Higher Inductive Types}

When constructing ever more complicated types, it is useful to have some
control over which terms are identified. For example we might want to construct
a quotient $A/_\sim$ where related terms are identified.

One way to do this is \emph{Higher Inductive Types}. Like inductive types, HITs
are constructed from generators, but while the generators of an inductive type
may only generate terms, the generators of a HIT may also generate identity
proofs. We will often use the language of Homotopy Theory and refer to the terms
and identity proofs in a HIT as ``points'' and ``paths'' respectively.

There is no reason to stop there! We could apply the same idea to give
generators for paths between paths, paths between paths between
paths and so on. In \autoref{sec/laws-hpt-noTrunc-noIndep} we will make use of
such ``higher paths'', but for now let us focus on a simple example.

\subsubsection{The Circle}
Utilizing the language of spaces we introduce a very simple HIT: the circle
$S^1$. This is a HIT with one point constructor $\base$ and one path constructor
$\sloop$. Its introduction and formation rules are:
\begin{equation*}
  \begin{array}[t]{c}
    \hline
    \Gamma \vdash S^1 \; \Type\\
  \end{array}
  \qquad
  \begin{array}[t]{c}
    \hline
    \Gamma \vdash \base : S^1\\
  \end{array}
  \qquad
  \begin{array}[t]{c}
    \hline
    \Gamma \vdash \sloop : \base =_{S^1} \base
  \end{array}
\end{equation*}

The circle also comes with a recursion principle~\footnote{The fully general
  induction principle requires a notion of dependent paths ``over'' $\sloop$. See section 6.2 in
  The Book~\cite{hottbook}}. Like the induction principle for lists,
a function out of the circle requires one term for each constructor. A point $p$
for the base, and a path from $p$ to itself for the loop.

\begin{equation*}
  \begin{array}[t]{c}
    \Gamma \vdash b : B \qquad \Gamma \vdash l : b =_B b\\
    \hline
    \Gamma \vdash rec_{S^1}~b~l : S^1 \ra B
  \end{array}
\end{equation*}

The computation rules postulate that the resulting function maps $\base$ to $b$
and applying it around $\sloop$ gives $l$.
\begin{equation*}
  \begin{array}[t]{c}
    \Gamma \vdash b : B \qquad \Gamma \vdash l : b =_B b\\
    \hline
    \Gamma \vdash rec_{S^1}~b~l \base \equiv b
  \end{array}
  \qquad
  \begin{array}[t]{c}
    \Gamma \vdash b : B \qquad \Gamma \vdash l : b =_B b\\
    \hline
    \Gamma \vdash \operatorname{loopComp} : \operatorname{cong} (rec_{S^1}~b~l) \sloop =_{b=b} l
  \end{array}
\end{equation*}
Note that the computation rule for $\sloop$ does not give a judgemental
equality, but rather postulates a term called $\operatorname{loopComp}$
witnessing the result.

\subsubsection{Synthetic Homotopy Theory}
An important use of HITs is synthetic homotopy theory~\cite{mortberg2020cubical,
  licata2015cubical}. By constructing spaces (or, more precisely cell
complexes) as HITs it is possible to formalize results of homotopy theory very
directly.

Here we recount a proof that the fundamental group of the circle is $\mathbb{Z}$
originally given by Licata and Shulman~\cite{licata2013circle} and by M\"ortberg
and Pujet~\cite{mortberg2020cubical}.This showcases the use of circle
recursion and also proves useful in \autoref{sec/elementary-hpt}.

The fundamental group of a space $X$ (with base point $x_0$) $pi_1(X, x_0)$ is
the group whose elements are loops at $x_0$ and whose operation is path
concatenation. In type theoretic terms we define the loop space (at $x_0$) $\Omega X$ as the
identity type $x_0 =_X x_0$. Path concatenation is simply the transitive
property of identity proofs.

The proof that $\pi_1(S^1, \base)$ has four parts:
\begin{enumerate}
  \item a function winding $: \Omega S^1 \ra \mathbb{Z}$
  \item a function intLoop $: \mathbb{Z} \ra \Omega S^1$
  \item a proof that winding and intLoop are mutual inverses
  \item a proof that either winding or intLoop is a group homomorphism (it
    suffices to show one direction)
\end{enumerate}
We limit ourselves to (1) to showcase circle recursion, but the full
details are spelled out in section 8.1 of The Book~\cite{hottbook}.

Since we cannot do recursion directly on $\Omega S^1$ we first construct the
covering space of the circle $\operatorname{helix}$. This is a family over $S^1$
which can be viewed as a function from $S^1$ to the universe, so we use circle
recursion to define:
\[
  \operatorname{helix} := rec_{S^1} ~ \mathbb{Z} ~ (\ua \operatorname{succEquiv})
\]
Where succEquiv is the equivalence $\mathbb{Z} \simeq \mathbb{Z}$ induced by the
successor function.

The winding number is computed by transporting in the helix. Given some $x:S^1$
and path $p : \base = x$, define:
\[
  \operatorname{encode}(p) := \transport^{\helix}~p~0
\]
and finally
\[
  \operatorname{winding} := \operatorname{encode} \base
\]

Transporting along $\sloop$ in helix is equivalent to transporting along
$\apcong \helix \sloop$ in the identity family, which by $\operatorname{loopComp}$
is identified with $\transport^{id}~(\ua~\operatorname{succEquiv})$. Computation
rules for $\ua$ give the final result: the successor function.

This technique -- making use of the recursion principle to define a function on
all of $S^1$ before specializing to $\base$ -- is known as an ``encode-decode''
style of proof.