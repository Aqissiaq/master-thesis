\section{Identity Types}\label{sec/identitytypes}

Given this notion of propositions as types, one of the things we may want to
propose (and prove) is the equality of two terms. That is, given two terms of
some type, how do we show that they are equal? Note that this is different from
the \emph{judgemental} equality discussed in \autoref{sec/typetheory}. [HOW EXACTLY?]

Since propositions are types and ``x is equal to y'' is a proposition, there
should be a corresponding type. Also, the truth of this proposition depends on
$x$ and $y$ (clearly ``2 is equal to 2'' should be different from ``2 is equal
to 3'') so the type should depend on $x$ and $y$ as well. But how should this
type be constructed? What are the terms of such a type?

The solution, proposed by Per Martin-Löf~\cite{ML75}, is an inductive family of dependent
types called the \emph{identity type}. For each type $A$ and pair of terms
$x,y:A$ we construct the identity type $x =_A y$ (the subscript may be dropped
when the type of $x$ and $y$ is clear). It has the following formation and introduction
rules~\cite{Rijke2019}:

\[
  \begin{array}[t]{c}
    \Gamma \vdash a:A\\
    \hline
    \Gamma, x:A \vdash a =_A x \; Type\\
  \end{array}
  \qquad
  \begin{array}[t]{c}
    \Gamma \vdash a:A\\
    \hline
    \Gamma \vdash refl_a : a =_A a
  \end{array}
\]
and an induction principle given by:
\[
  \begin{array}{c}
    \Gamma \vdash a:A \qquad \Gamma, x:A, p:a =_A x \vdash P (x,p) Type\\
    \hline
    \Gamma \vdash J_a : P (a, refl_a) \rightarrow \Pi_{x:A}\Pi_{p:a=_Ax}P
    (x, p)
  \end{array}
\]

This is astonishingly simple! The identity type has one constructor: $refl_{\_}$,
and in order to use its terms $p : x =_A y$ it is enough to know the case when $
\equiv y$ and $p \equiv refl$.

Despite the few ingredients, identity types exhibit a great deal of (admittedly
expected) structure. For example, the identity type $=_A$ on some type $A$ is an
equivalence relation. [TYPE THEORETICALLY? WHAT DO WE CALL THIS?] It is clearly
reflexive ($x =_A x$ is inhabited by $refl_x$), but it is also symmetric and
transitive. Given proofs $p : x = y$ and $q : y = z$, let us denote the
symmetric proof by $p^{-1} : y = x$ and the result of transitivity $p \dot q : x
= z$.

Given a term $a:A$ the J-rule lets us inhabit a type $P(x,p)$ by providing a
term of $P(a, refl_a)$. This reduces the task of showing symmetry and
transitivity to the cases when the paths are all $refl$. The inverse $refl^{-1}$
is again $refl$ and the composition $refl \cdot refl$ is also $refl$.

%% a more technical, arguably less clear version
% For a fixed $a : A$, symmetry is expressed by $P(x,p) := x =_A a$. Since $P(a,
% refl_a) := a =_A a$ is inhabited by $refl_a$, $J_a$ gives an inverse for any $x
% : A$ and $p : a =_A x$. Denote this inverse by $p^{-1}$.

\subsubsection{UIP and The Groupoid Structure of Types}

A question one might ask is ``can there be more than one proof of identity?''.
The affirmative answer is a property known as \emph{Uniqueness of Identity Proofs} (UIP). A
type $A$ satisfies UIP if for any $x~y : A$ and $p~q : x =_A y$ the type $p =_{x
= y} q$ is inhabited\footnote{In HoTT we say A \emph{is a set}.}. Now the
question of uniqueness can be posed as ``does UIP hold for every type?'' In 1995
Hofman and Streicher~\cite{Hofman1998} showed that the answer is ``no'' by
constructing a model in which it fails to hold.

What, then, are the relationships between these proofs? Hofman and Streicher's
model give and answer here too. In it, types are \emph{groupoids} and the
identity type $x =_A y$ is modeled by $Hom_A(x, y)$. In addition to the
properties we have already seen, they show that composition is associative and
respects units and inverses. That is, for proofs $p : x = y$, $q : y = z$, $r :
z = w$ the following types are all inhabited:
\[(p \cdot q) \cdot r =_{x = w} p \cdot (q \cdot r)\]
\[p \cdot refl_y = p\]
\[refl_x \cdot p = p\]
\[p \cdot p^{-1} =_{x = x} refl_x\]
\[p^{-1} \cdot p =_{y = y} refl_y\]

This \emph{groupoid structure} of proofs will be very important in \autoref{ch/formalization}.
