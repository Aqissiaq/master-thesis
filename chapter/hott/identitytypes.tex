\section{Identity Types}\label{sec/identitytypes}

Given this notion of propositions as types, one of the things we may want to
propose (and prove) is the equality of two terms. That is, given two terms of
some type, how do we show that they are equal? Note that this is different from
the \emph{judgemental} equality discussed in \autoref{sec/typetheory}. [HOW EXACTLY?]

Since propositions are types and ``x is equal to y'' is a proposition, there
should be a corresponding type. Also, the truth of this proposition depends on
$x$ and $y$ (clearly ``2 is equal to 2'' should be different from ``2 is equal
to 3'') so the type should depend on $x$ and $y$ as well. But how should this
type be constructed? What are the terms of such a type?

The solution, proposed by Per Martin-Löf~\cite{ML75}, is an inductive family of dependent
types called the \emph{identity type}. For each type $A$ and pair of terms
$x,y:A$ we construct the identity type $x =_A y$ (the subscript may be dropped
when the type of $x$ and $y$ is clear). It has the following formation and introduction
rules~\cite{Rijke2019}:

\[
  \begin{array}[t]{c}
    \Gamma \vdash a:A\\
    \hline
    \Gamma, x:A \vdash a =_A x \; Type\\
  \end{array}
  \qquad
  \begin{array}[t]{c}
    \Gamma \vdash a:A\\
    \hline
    \Gamma \vdash refl_a : a =_A a
  \end{array}
\]
and an induction principle given by:
\[
  \begin{array}{c}
    \Gamma \vdash a:A \qquad \Gamma, x:A, p:a =_A x \vdash P (x,p) Type\\
    \hline
    \Gamma \vdash J_a : P (a, refl_a) \rightarrow \Pi_{x:A}\Pi_{p:a=_Ax}P
    (x, p)
  \end{array}
\]

This is astonishingly simple! The identity type has one constructor: $refl_{\_}$,
and in order to use its terms it is enough to know how to use $refl$.

[MAYBE SOMETHING ABOUT THE GROUPOID STRUCTURE AND UIP]

One way to make sense of identity types is through the homotopy theory. With
this interpretation a term of $x =_A y$ is like a path in $A$ from $x$ to $y$.
In fact the collection of all such paths is itself a space (and thus a type):
the path space. Additionally there may be paths between paths, paths between
paths between paths and so on. These higher paths are the eponymous
``homotopies'' and provide a rich field of study on their own. [REVISIT]