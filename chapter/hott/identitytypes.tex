\section{Identity Types}\label{sec/identitytypes}

Given this notion of propositions as types, one of the things we may want to
propose (and prove) is the equality of two terms. That is, given two terms of
some type, how do we show that they are equal? Note that this
\emph{propositional} equality is different from the \emph{judgemental} equality
discussed in \autoref{sec/typetheory}. While judgemental equality is part of the
inference rules of the type theory, the propositional equality of two terms is a
type that can be inhabited and whose terms can be used. 

Since propositions are types and ``x is equal to y'' is a proposition, there
should be a corresponding type. Also, the truth of this proposition depends on
$x$ and $y$ (clearly ``2 is equal to 2'' should be different from ``2 is equal
to 3'') so the type should depend on $x$ and $y$ as well. But how should this
type be constructed? What are the terms of such a type?

The solution, proposed by Per Martin-Löf~\cite{ML75}, is an inductive family of dependent
types called the \emph{identity type}. For each type $A$ and pair of terms
$x,y:A$ we construct the identity type $x =_A y$ (the subscript may be dropped
when the type of $x$ and $y$ is clear). It has the following formation and introduction
rules~\cite{Rijke2019}:

\[
  \begin{array}[t]{c}
    \Gamma \vdash a:A\\
    \hline
    \Gamma, x:A \vdash a =_A x \; \Type\\
  \end{array}
  \qquad
  \begin{array}[t]{c}
    \Gamma \vdash a:A\\
    \hline
    \Gamma \vdash \refl_a : a =_A a
  \end{array}
\]
and an elimination rule (called the induction principle or simply the J-rule) given by:
\[
  \begin{array}[t]{l}
    \Gamma \vdash a:A\\
    \Gamma, x:A, p:a =_A x \vdash P~x~p \Type\\
    \Gamma, a:A \vdash J_a : P~a~\refl_a\\
    \hline
    \Gamma, x:A, p:a=_Ax \vdash P~x~p
  \end{array}
\]

This is astonishingly simple! The identity type has one constructor: $\refl_{\_}$,
and in order to use its terms $p : x =_A y$ it is enough to know the case when $x
\equiv y$ and $p \equiv \refl$.

Despite the few ingredients, identity types exhibit a great deal of (admittedly
expected) structure. For example, the identity type $=_A$ on some type $A$ is an
equivalence relation. It is clearly reflexive ($x =_A x$ is inhabited by
$\refl_x$), but it is also symmetric and transitive. Given proofs $p : x = y$
and $q : y = z$, let us denote the symmetric proof by $p^{-1} : y = x$ and the
result of transitivity $p \cdot q : x = z$.

Given a term $a:A$ the J-rule lets us inhabit a type $P(x,p)$ by providing a
term of $P(a, \refl_a)$. This reduces the task of showing symmetry and
transitivity to the cases when the paths are all $\refl$. The inverse $\refl^{-1}$
is again $\refl$ and the composition $\refl \cdot \refl$ is also $\refl$.

%% a more technical, arguably less clear version
% For a fixed $a : A$, symmetry is expressed by $P(x,p) := x =_A a$. Since $P(a,
% \refl_a) := a =_A a$ is inhabited by $\refl_a$, $J_a$ gives an inverse for any $x
% : A$ and $p : a =_A x$. Denote this inverse by $p^{-1}$.

Armed with the J-rule many useful functions on identity types can be
constructed. We mention two closely related functions:

The first is $\apcong~f$ (called $ap_f$ in \cite{hottbook, lemma 2.2.1}) which for some
function $f : A \ra B$ maps identity proofs $a = a'$ in $A$ to identity proofs
$f~a = f~a'$ in $B$. By the J-rule we may assume that the path in question is
$\refl_x$ and define $\apcong~f~\refl_x := refl_{f~x}$.

The second is a dependent version of $\apcong$ called $\transport$. Given a
family $P$ over $A$ and an identification $p : a =_A a'$, $\transport^P~p$ gives
a function $P~a \ra P~a'$. In the case where $p$ is $refl_{x}$ it is the
identity function $P~x \ra P~x$.


\subsubsection{UIP and The Groupoid Structure of Types}
A question one might ask is ``can there be more than one proof of identity?''.
The negative answer is a property known as \emph{Uniqueness of Identity Proofs} (UIP). A
type $A$ satisfies UIP if for any $x~y : A$ and $p~q : x =_A y$ the type $p =_{x
= y} q$ is inhabited\footnote{In HoTT we say A \emph{is a set} or has H-level 0.}. Now the
question of uniqueness can be posed as ``does UIP necessarily hold for every type?'' In 1995
Hofman and Streicher~\cite{Hofman1998} showed that the answer is ``no'' by
constructing a model in which it fails to hold.

So an identity is not merely a type which either does or does not have an
inhabitant, but does it have more structure?
Hofman and Streicher's model give an answer here too. In it, types are \emph{groupoids} and the
identity type $x =_A y$ is modeled by $Hom_A(x, y)$. In addition to the
properties we have already seen, they show that composition is associative and
respects units and inverses. That is, for proofs $p : x = y$, $q : y = z$, $r :
z = w$ the following types are all inhabited:
\[(p \cdot q) \cdot r =_{x = w} p \cdot (q \cdot r)\]
\[p \cdot \refl_y =_{x = y} p\]
\[\refl_x \cdot p_{x = y} = p\]
\[p \cdot p^{-1} =_{x = x} \refl_x\]
\[p^{-1} \cdot p =_{y = y} \refl_y\]

This \emph{groupoid structure} of identity proofs will be very important in \autoref{ch/formalization}.
