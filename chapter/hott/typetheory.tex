\section{(Dependent) Type Theory}
\begin{enumerate}
  \item (dependent) types in computer science
  \item type theories in math/foundations
  \item the formal stuff
\end{enumerate}

Types are a familiar concept to the computer scientist. We are used to working
with data, and this data often has a \emph{data type} either explicitly or
implicitly. For example, \texttt{42} is an \texttt{int}, \texttt{'c'} is a
\texttt{char}, and \texttt{['a','b','c']} is a list of \texttt{char}s (henceforth
denoted \texttt{[char]}). We call
\texttt{int}, \texttt{char} and \texttt{[char]} \emph{types} and
\texttt{42}, \texttt{'c'}, \texttt{['a','b','c']} \emph{terms} of those types.
While this is a good basis for intuition, Type Theory (tm) is a bit different.

However, let us stick with the programming intuition to introduce a less
familiar concept: \emph{dependent} types. First, note that one of the types in
the previous paragraph is a bit different than the others: \text{['a','b','c']} is a
list \emph{of \texttt{char}s}. Similarly we could have lists of \texttt{int}s,
lists of \texttt{float}s or even lists of lists! Clearly ``lists'' comprises
many different types, depending on the type of their elements. We could call
\texttt{list} a family of types \emph{parametrized} by types. Such a family is
actually a whole collection of types -- one for each other type we can make
lists of.
Dependent types extend this idea by allowing families to be parametrized by
terms. Then we can create new and exciting types like \texttt{Vec 3} and
\texttt{Vec 4} -- the types of 3- and 4-dimensional vectors. Again \texttt{Vec} is
actually a whole collection of types -- one for each integer!

We now leave the familiar world of programming behind and venture in to the spooky
(but exciting) world of foundational mathematics.
\begin{equation}
  \begin{array}{c}
    \Gamma \vdash a : A \qquad \Gamma \vdash f : A \rightarrow B\\
    \hline
    \Gamma \vdash f(a) : B
  \end{array}
  \label{rule:example}
  \end{equation}

In this new and wondrous world, a type theory is a system of \emph{inference
  rules} like \ref{rule:example} that can be used to make \emph{derivations}.