\section{Types and Propositions (and spaces?)}
\begin{enumerate}
  \item types represent propositions (and spaces)
  \item implication and simple and/or ($\rightarrow, \times, +$)
  \item quantifiers and dependent types (fibers) ($\Sigma, \Pi$)
\end{enumerate}

In this section we consider an important interpretation of type theory: the
Howard-Curry Isomorphism (which isn't an isomorphism, but we're not going into
those details).

Under this ``isomorphism'' types are identified with logical propositions, and
terms with proofs of those propositions. The most basic example is the unit type
$\top$. It has exactly one term, denoted $tt$, and we interpret it as ``true''.

Another important type is the empty type $\bot$ which has no terms and is
interpreted as ``false''. This might seem strange, but will make a lot of sense
once we start using it.

Let us make some more elaborate propositions. For example given the types (and
hence propositions) $A$ and $B$ what would it mean to prove $A \land B$? Well if
both $A$ and $B$ are true, we should be able to give a proof of $A$ \emph{and}
and proof of $B$. But since proofs are terms of the corresponding type, this is
the same as having terms $a : A$ and $b : B$. To keep track of both, lets form
the ordered pair $(a, b)$. This is precisely an element of the product type $A
\times B$! Hence this product type represents the proposition $A \land B$, since
its terms correspond exactly to proofs of $A$ and $B$.

As a sanity check, consider the truth table of $A \land B$ alongside the terms of $A
\times B$ using $\top$ and $bottom$ to represent true and false. $A \land B$ is
true when both $A$ and $B$ are true, and similarly $A \times B$ is inhabited
exactly when both $A$ and $B$ are inhabited. (This doesn't really work because
I have not given a true $\leftrightarrow$ inhabited correspondence)

\begin{table}[h!]
\centering
\subfloat[logic]{
\begin{tabular}{|l|l|l|}
\hline
$A$ & $B$ & $A \land B$ \\
\hline
false   & false   & false\\
false   & true   & false\\
true   & false   & false\\
true   & true   & true\\   
\hline
\end{tabular}}%
\hspace{.25\linewidth}
\subfloat[types]{
\begin{tabular}{|l|l|l|}
\hline
$A$ & $B$ & $A \times B$ \\
\hline
$\bot$   & $\bot$   & $\bot$\\
$\bot$   & $\top$   & $\bot$\\
$\top$   & $\bot$   & $\bot$\\
$\top$   & $\top$   & $\top$\\       
\hline
\end{tabular}}
\end{table}
