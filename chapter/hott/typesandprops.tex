\section{Propositions as Types}
\begin{enumerate}
  \item types represent propositions (and spaces)
  \item implication and simple and/or ($\rightarrow, \times, +$)
  \item quantifiers and dependent types (fibers) ($\Sigma, \Pi$)
\end{enumerate}

In this section we consider an important interpretation of type theory: the
Howard-Curry Isomorphism (which isn't an isomorphism, but we're not going into
those details).

Under this ``isomorphism'' types are identified with logical propositions, and
terms with proofs of those propositions. This means we can consider a
proposition ``true'' (or at least ``proved'') if we can construct a term of the
corresponding type.

Two very simple types are the empty type $\bot$ which has not terms, and the
unit type $\top$ which has one term denoted by $\mathbf{1}$. [MAYBE INTRODUCE
THE TYPES FIRST]

Under the ``types as propositions'' interpretation, $\bot$ represents
\emph{false}. The type has no terms so there are no proofs of ``false'', just
like we would expect from a sound system. (Of course this alone does not prove
our type theory sound.) Similarly, $\top$ represents \emph{true}. It always has
a proof: $\mathbf{1}$.

Let us make some more elaborate propositions. For example given the types (and
hence propositions) $A$ and $B$ what would it mean to prove $A \land B$? Well if
both $A$ and $B$ are true, we should be able to give a proof of $A$ \emph{and}
and proof of $B$. But since proofs are terms of the corresponding type, this is
the same as having terms $a : A$ and $b : B$. To keep track of both, lets form
the ordered pair $(a, b)$. This is precisely an element of the product type $A
\times B$! Hence this product type represents the proposition $A \land B$, since
its terms correspond exactly to proofs of $A$ and $B$.

As a sanity check, consider the truth table of $A \land B$~(\autoref{fig:and-truth-table}) alongside the terms of $A
\times B$~(\autoref{fig:product-type-table}) using $\top$ and $\bot$ to represent true and false. $A \land B$ is
true when both $A$ and $B$ are true, and similarly $A \times B$ is inhabited
exactly when both $A$ and $B$ are inhabited.

\begin{table}[h!]
\centering
\subfloat[logic]{
\begin{tabular}{|l|l|l|}
\hline
$A$ & $B$ & $A \land B$ \\
\hline
false   & false   & false\\
false   & true   & false\\
true   & false   & false\\
true   & true   & true\\   
\hline
\end{tabular}
\label{fig:and-truth-table}}%
\hspace{.25\linewidth}
\subfloat[types ``()'' meaning there are no terms of this type]{
\begin{tabular}{|l|l|l|}
\hline
$A$ & $B$ & $A \times B$ \\
\hline
$\bot$   & $\bot$   & $()$\\
$\bot$   & $\top$   & $()$\\
$\top$   & $\bot$   & $()$\\
$\top$   & $\top$   & $(\mathbf{1}, \mathbf{1})$\\       
\hline
\end{tabular}
\label{fig:product-type-table}}
\end{table}

As another example, what does it mean to prove an implication $A \rightarrow B$?
One reasonable answer is that given a proof of $A$, I can produce a proof of
$B$. In terms of types, that means a way to produce a term of type $B$ given a
term of type $A$, which is exactly a function from $A$ to $B$! Finally, note
that logical ``or'' is represented by the sum type (disjoint union) $A + B$.

[NOTE: this results in a \emph{constructive} logic (good)]

We have the basic building blocks of propositional logic, but what about
first-order logic with $\exists$ and $\forall$? This is where our dependent
types come in handy.

First, let us note that a predicate on a variable is a lot like a dependent
type. If simple types can be interpreted as propositions, and a predicate on
some variable is a proposition that \emph{depends} on a variable, then it stands
to reason that a predicate can be represented with a dependent type. As such, we
may view a term of the type $B(x)$ as a proof that $B$ holds for the term $x$.
[WHATEVER THAT MEANS, LOL]

Extending this thinking to quantifiers and considering what it means to provide
a proof, a proof of $\exists x. P(x)$ should consist of some $x:A$ [CHEATING IN
THE A] and a proof that $P$ is true of $x$. Such a pair is a term of a type we have seen
before: the dependent pair $\Sigma_A P$. Note that this term actually contains
\emph{more} data than just asserting $\exists x. P(x)$ -- it gives us an $x$.

Similarly, a proof of $\forall x. P(x)$ can be seen as an assertion that
whatever $x:A$ you give me, I can show that produce a proof (term) of $P(x)$. We
use $\Pi_A P$ to represent this quantification. Note that both of these
constructions quantify over some base type $A$, not over ``every $x$ in
the universe,'' whatever that means. [REVISIT]