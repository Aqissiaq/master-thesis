\chapter{Conclusion}

In this thesis we have constructed and outlined an implementation of homotopical
patch theory in Cubical Agda. The implementation makes use of higher inductive
types and univalence, and since the cubical model imbues univalence with
computational meaning we are able to show that models of the theory behave as
expected -- at least for simple examples.

A full exploration of the behavior is (at the time of writing) limited by two
factors: the efficiency of typechecking for complicated terms, and the fact that
Cubical Agda does not fully reduce \texttt{transp} and \texttt{hcomp} for
indexed families of types.

The former means that it is computationally expensive (and time consuming) to
verify the behavior of the implementation, while the latter makes it impossible
to compute results for the more complicated models. In particular for
compositions of patches (using \texttt{hcomp}) when modeling the repository as a
vector of strings (and indexed family of types).

Additionally, the first chapter contains an exposition of HoTT and (cubical)
Agda and the second a survey of some approaches to a theory of VCS.

\section{Future Work}

There are two main avenues for future work. Firstly on the formalization of HPT
and secondly on other type-theoretic approaches like the one discussed in
\autoref{sec:attempt}.

The HPT formalization also permits two directions of further inquiry. One is to
implement more of the original paper. In particular the \texttt{indep} law for
the patch theory with laws and all patch laws for the theory in section 6 are
missing. Their inclusion would require a different way to map higher-dimensional
paths into the universe which is guaranteed to terminate.

The other is to work towards more computational results.
Specifically we are limited by indexed families, and further work in Cubical
Agda's normalization would lead to more results ``for free''. It would also be
interesting to look at the computation of \texttt{opt}, as it requires some
notions of sub-singletons~\cite{Angiuli2016} and is very time consuming at the
time of writing.

Additionally, it might be possible to obtain more results by writing a more
direct implementation. In May of 2022 M\"ortberg and Ljungstr\"om succeeded in
computing the Brunerie number in Cubical Agda in a matter of seconds [cite
axel's blog post when it shows up?] - where previous attempts had resulted in
programs that ran for over 90 hours without a result~\cite{mortberg2018}. The
new proof achieves this by providing a more direct construction [citation
needed] and avoiding computationally expensive equivalences (in particular $S^3
\simeq S^1 \ast S^1$). Their development suggests that more results may be
achieved by identifying the computationally expensive aspects of the current
construction, and avoiding these with more direct alternatives.

Other type theoretic approaches could also be investigated. While HPT provides
an elegant encapsulation of groupoid properties and separation of theories from
models, it does not provide tools to reason about the semantics of patches and
operations on them. In particular, a formal theory could incorporate merging and
its properties. Another direction for expansion would be a more modular theory
in which different patch theories (e.g for text files and integers) could be
combined in a principled way to form something like a directory.