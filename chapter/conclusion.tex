\chapter{Conclusion}\label{ch/conclusion}

In this thesis we have constructed and outlined an implementation of homotopical
patch theory in Cubical Agda\footnote{Available here:
  \url{https://github.com/Aqissiaq/hpt-experiments}}.
The implementation makes use of higher inductive
types and univalence, and since the cubical model imbues univalence with
computational meaning we are able to show that models of the theory behave as
expected -- at least for simple examples.

A full exploration of the behavior is (at the time of writing) limited by two
factors: the efficiency of typechecking for complicated terms, and the fact that
Cubical Agda does not fully reduce \texttt{transp} and \texttt{hcomp} for
inductive families of types.

The former means that it is computationally expensive (and time consuming) to
verify the behavior of the implementation, while the latter makes it impossible
to compute results for the more complicated models. In particular for
compositions of patches (using \texttt{hcomp}) when modeling the repository as a
vector of strings since it is an inductive family of types.

This chapter discusses the implementation and results of
\autoref{ch/formalization} and outlines directions for future work on homotopical
patch theory.

\section{Discussion}
Cubical Agda provides a good setting for formalizations relying on univalence,
function extensionality and higher inductive types. Implementation of the
ideas of HPT was direct, avoiding complicated elimination rules associated with
HITs and univalence~\footnote{Compare the length of just the contractibility
proof in \url{https://github.com/dlicata335/hott-agda/blob/master/programming/PatchWithHistories3.agda}
to our one-liner.}.

However, a few complications were encountered. In particular the independence
patch law in ``a patch theory with laws'' is a glaring omission. While it is
easy to state, it does not play well with our technique -- adapted from the Cubical library's set
truncation -- for mapping out of higher path constructors. This is not a
fundamental limitation, and may have solutions, but was a hurdle.

The other omission is the patch laws of ``a patch theory with richer contexts''.
In this case the issues arose already trying to define \texttt{Interpreter}. The
cases for patch laws should encode the fact that \texttt{replay} respects these
patch laws, but since repositories are already indexed by a HIT of histories,
the resulting type becomes very complicated and I was unable to state it
correctly.

We might also expect that \emph{any} square will do, since all four corners are
singletons and all singletons are equivalent. However, this does not appear to
pass Cubical Agda's typechecking because the sides must \emph{definitionally}
agree with the mapping of points and patches.

The computational results are promising, but also reveal the current limits of
the Cubical Agda implementation.
There are two distinct issues: firstly the issue of \texttt{transp} and
\texttt{hcomp} over inductive families which simply does not reduce, and
secondly an issue of performance.

The performance limitations are also visible in two distinct places. First,
recall the elementary computations in \autoref{subsec:elementary-results}. A
detailed performance analysis is outside the scope of this thesis, but it is
noteworthy that computing 1000 successor functions takes significant time.

Second, the results of \texttt{optimize} appear to be very slow. In fact, the
author has been unable to compute the result of applying any optimized patch due
to memory constraints. As noted in \autoref{sec:results} the results themselves
are not in question, since \texttt{optimize} also produces a proof that the
resulting patch is equal to the original, but the resource use is interesting
nevertheless.

Possible causes include the trick used to define \texttt{opt} of the \texttt{noop}
constructor and the transport along the long sequence of paths constituting
\texttt{e}.

\section{Future Work}

There are two main avenues for future work. Firstly on the formalization of HPT
and secondly on other type-theoretic approaches like the one discussed in
\autoref{sec:attempt}.

The HPT formalization also permits two directions of further inquiry. One is to
implement more of the original paper. In particular the \texttt{indep} law for
the patch theory with laws and all patch laws for the theory in section 6 are
missing. Their inclusion would require a different way to map higher-dimensional
paths into the universe which is guaranteed to terminate.

Additionally, Angiuli et al. implement an alternative version of the patch
theory with laws in which \texttt{R} is set-truncated by adding another
constructor. This is used as an alternative to the contractibility of the target
type when defining \texttt{opt} and may offer a solution to the termination
problems of the independence patch law.

The other is to work towards more computational results.
Specifically we are limited by inductive families, and further work in Cubical
Agda's normalization would lead to more results ``for free''~\footnote{This is
  an active area of work for Cubical Agda, see for example
  \url{https://github.com/agda/agda/issues/3733\#issuecomment-903581647},
  \url{https://github.com/agda/cubical/pull/309} and section 3.2.4 of \cite{vezzosi2021cubical}}.

It would also be interesting to look at the computation of \texttt{opt}, as it requires some
notions of sub-singletons~\cite{Angiuli2016} and is very time consuming at the
time of writing.

Additionally, it might be possible to obtain more results by writing a more
direct implementation. In May of 2022 M\"ortberg and Ljungstr\"om succeeded in
computing the Brunerie number in Cubical Agda in a matter of
seconds~\cite{ljungstrom2022} -- where previous attempts had resulted in
programs that ran for over 90 hours without a result~\cite{mortberg2018}. The
new proof achieves this by providing a more direct construction and avoiding
computationally expensive equivalences (in particular $S^3 \simeq S^1 \ast
S^1$). Their development suggests that more results may be achieved by
identifying the computationally expensive aspects of the current construction,
and avoiding these with more direct alternatives.

Other type theoretic approaches could also be investigated. While HPT provides
an elegant encapsulation of groupoid properties and separation of theories from
models, it does not provide tools to reason about the semantics of patches and
operations on them. In particular, a different model could incorporate merging and
its properties. Another direction for expansion would be a more modular theory
in which different patch theories (e.g for text files and integers) could be
combined in a principled way to form something like a directory.
An attempt at an alternative approach is included in \autoref{sec:attempt}.