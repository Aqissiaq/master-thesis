\section{Background}

This section introduces the terminology and overall structure of version control
systems for use in the ensuing work. We focus on a patch-theoretic view in the
style of Darcs~\cite{Darcs, Lynagh2006, Sittampalam2005,
  jacobson2009formalization} with groupoid properties as in Homotopical Patch
Theory~\cite{Angiuli2016}.

\subsection{Basic Ingredients}
%repos and patches
Version control systems keep track of \emph{data}. In the most common use case this is
one or more files containing lines of text, but we do not need to specialize.
The basic ingredients of a VCS are \emph{repositories} and \emph{patches}. A
repository contains the data we are keeping track of, and a patch records a
change made to the repository.

\subsection{(Groupoid) Properties of Patches}
In Patch Theory a repository is a collection of patches, from which the
data can be reconstructed. At any given time the specific collection of patches
describe a \emph{repository state}.

A patch records a change like ``add the line $l$ to the file $f$''. This
patch is nonsensical if there is no file $f$, so each patch has a domain
\emph{context} in which it can be applied. The context is a repository state,
and a patch can be \emph{applied} to a repository in the appropriate state.
Each patch also has codomain context which is the state it leaves the repository
in. In this section we denote a patch $P$ with domain $x$ and codomain $y$ by
$_xP_y$. We may leave out the contexts if they are not important.

For the theory to be useful we should be able to apply more than one patch to a
repository. For this purpose there is patch composition. Given two patches
$_xP_y$ and $_yQ_z$ with matching contexts there is a composite patch $_xP \cdot
Q_z$ which is meant to represent ``apply $P$ and then apply $Q$''.

In any given patch theory~\footnote{While Patch Theory refers to the study of
VCS as repositories characterized by collections of patches, \emph{a} patch
theory denotes a specific collection of patches and laws they obey.} we have
some collection of possible patches and laws they must obey. In the context of
this work we assume a few basic laws for all patch theories:

\begin{enumerate}
\item For each context $x$ there exists a ``do-nothing'' patch $_xN_x$ such that
  for any patches $_xP_y$:
  \[_xN \cdot P_y \,=\, _xP_y \, =\, _xP \cdot N_y\]
  ie. $R$ is both a left and right unit for composition.
\item Associativity of patch composition. Given patches $_xP_y$, $_yQ_z$ and
  $_zR_w$:
  \[_x(P \cdot Q) \cdot R_w \, = \, _xP \cdot (Q \cdot R)_w\]
\item For each patch $_xP_y$ there exists an inverse $_yP^{-1}_x$ such that:
  \[_xP \cdot P^{-1}_x \,=\, _xN_x\]
\end{enumerate}

These laws are not arbitrary. In fact they express the assumption that a the
repository states and patches between them form a \emph{groupoid}. This will be
especially important for \autoref{subsec:hpt} and the following implementation
in \autoref{ch/formalization}.

\subsection{Merging}

Another common feature of VCSs is \emph{branching}. Branching occurs in
distributed systems when two users of the repository have different repository
states. If the users wish to reconcile their states, they perform a
\emph{merge}.

In our setting merge is a function that takes a span of patches $(_0P_x,
_0Q_y)$ (here $0$ is some shared base context) and produces a cospan $(_xP'_z ,
_yQ'_z)$ (\autoref{fig:merge}). 

\begin{figure}
\begin{centering}
\begin{tikzcd}
                            & 0 \arrow[ld, "P"'] \arrow[rd, "Q"] &                            \\
x \arrow[rd, "P'"', dashed] &                                    & y \arrow[ld, "Q'", dashed] \\
                            & z                                  &                           
\end{tikzcd}
\caption{Merging $(P,Q)$}
\label{fig:merge}
\end{centering}
\end{figure}

For merge to be well behaved we might also want some other properties. Say merge
is \emph{symmetric} if \[merge(P,Q) = (P', Q') \implies merge(Q, P) = (Q', P')\]
and that merge is \emph{reconciling} if \[merge(P,Q) = (P',Q') \implies P \cdot
P' = Q \cdot Q'\]

Note that the groupoid laws imply that there always exists a merge of two
patches which is both symmetric and reconciling: take $z = 0$, $P' = P^{-1}$ and
$Q' = Q^{-1}$. This is the cospan that simply undoes the changes in both users' repositories and it
is not very interesting. We call this the \emph{trivial merge}.