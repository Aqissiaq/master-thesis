% This section header needs to be moved around if we change the order
% What's the canonical solution here?
\section{Prior work / Theoretical Approaches to VCS}

In this section we introduce some proposed theoretical models of version control
systems. These are the patch theory of Darcs~\cite{Darcs, Lynagh2006}, the
categorical approach of Mimram and Di Giusto~\cite{Pijul, Categorical2013} and
Angiuli et al.'s ``homotopical patch theory''~\cite{Angiuli2016} (hpt).

% START OF PROPER SUBSECTION
\subsection{Patch Theory (Darcs)}
Here we discuss several proposed formalisms for a the patch theory employed by
Darcs~\cite{Darcs}. \cite{Lynagh2006, Sittampalam2005, Dagit2009} all attempt to
describe Darcs' patch theory. (focus on Lynagh, I think)

% Lynagh's Algebra of Patches
\textcite{Lynagh2006} proposes an ``algebra of patches'' as a theoretical
basis for the Darcs~\cite{Darcs} version control system.

In this model a repository state is a set of updates (called \emph{patches}, but
we want to avoid that ambiguity) and a patch is a change to this set. For
example pulling the repository $\{c\}$ into the repository $\{a,b\}$ results in
a new repository $\{a,b\} \cup \{c\} = \{a,b,c\}$.

Patches are only applicable to one repository state, and result in a new state.
If they are compatible, we may string them together into a \emph{patch sequence}.
Denoting the previous example patch by $P$ and the ``do-nothing'' patch by $Id$
we have $\{a,b\}P\{a,b,c\}Id\{a,b,c\}$ -- pulling $\{c\}$ followed by doing
nothing. The repository state may be omitted from sequences.

Finally a notion of \emph{commutation} of patches is defined. We say the patch sequence $AB$
commute if there are patches $A'$ and $B'$ such that the following square commutes:

\begin{figure}[h!]
\begin{center}
  \begin{tikzcd}
  \bullet \arrow[rr, "B"] &  & \bullet \\
   &&\\
  \bullet \arrow[uu, "A"'] \arrow[rr, "B'"', dashed] &  & \bullet  \arrow[uu, "A'", dashed]
  \end{tikzcd}
\end{center}
\caption{Commuting patches}
\label{fig:darcs-commuting-patches}
\end{figure}

and write $AB \comm B'A'$. Note that the initial and final contexts (bottom left and top right,
respectively) are the same, but the intermediary contexts need not be.

There are four axioms for patches and commutation:
\begin{enumerate}
  \item Commutativity (lol) (3.1): $AB \comm B'A' \iff B'A' \comm AB$
  \item Invertibility (3.2): for each $A$ there is an $A^{-1}$ s.t
    $AA^{-1}=A^{-1}A=Id$
  \item Inv-cong (3.3): $AB \comm B'A' \iff A^{-1}B' \comm BA'^{-1}$. (we can
    start in the top left corner of \autoref{fig:darcs-commuting-patches} if we want)
    \label{axiom:darcs-inv-cong}
  \item Circular (3.5/6): performing all pairwise commutations in a sequence
    gets us back to the beginning (or, a horrible equation)
\end{enumerate}

These axioms allow us to define some useful operations on repositories. For
example, given a span $\{a\} \xleftarrow{A} \bullet \xrightarrow{B} \{b\}$ we
may want to incorporate the results of both patches to get $\{a,b\}$. We call
this operation ``merge'' and proceed in three steps:
\begin{enumerate}
\item by invertibility, we can find a patch $\{a\}A^{-1} \bullet$
\item now that we have a sequence $A^{-1}B$, we commute it to get the sequence $B'{A^{-1}}'$
\item define merge($A, B$) to be the sequence $AB'$.
\end{enumerate}
This process is shown in \autoref{fig:darcs-merge}.

\begin{figure}
\begin{center}
\begin{tikzcd}
\bullet \arrow[dd, "A"'] \arrow[rr, "B"]  &  & \{b\} \\
&  & \\
\{a\} \arrow[uu, "A^{-1}"', dashed, bend right] \arrow[rr, "B'"', dashed] & & \{a,b\} \arrow[uu, "{A^{-1}}'", dashed]
\end{tikzcd}
\end{center}
\caption{Merging A and B by commutation}
\label{fig:darcs-merge}
\end{figure}

Another useful operation on repositories is ``cherry picking''. Cherry picking
is the act of pulling some, but not all, patches from one repository into another.
Consider the patch sequence $\{\}A\{a\}B\{a,b\}C\{a,b,c\}$ and a repository
$\{a\}$. We want to incorporate the changes in $C$, but not the ones in $B$, but
naively combining applying $C$ does not work, since it is only applicable to the
context $\{a,b\}$. The solution is to commute $BC \leftrightarrow C'B'$ (\autoref{fig:darcs-cherry-pick}) to
obtain $C'$ with the desired endpoints.

\begin{figure}
\begin{center}
\begin{tikzcd}
{\{a,b\}} \arrow[rr, "C"]&  & {\{a,b,c\}}\\
&  &\\
\{a\} \arrow[uu, "B"] \arrow[rr, "C'"', dashed] &  & {\{a,c\}} \arrow[uu, "B'"', dashed]
\end{tikzcd}
\end{center}
\caption{Commutation for cherry picking}
\label{fig:darcs-cherry-pick}
\end{figure}

Problem: we cannot always commute patches, and Darcs does not have a great
solution here.