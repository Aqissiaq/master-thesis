\section{Another Type-Theoretic Approach}

In this section we explain a different approach to modeling VCSs in homotopy
type theory based on a specific class of HITs called co-equalizers.

The author has not succeeded in implementing this approach, but here it is
anyway [LOL].

\subsection{Approach}
Basically:
\begin{enumerate}
\item requirements
  \begin{itemize}
  \item repo - accurately represents contents
  \item patch - applicable in a context, groupoid structure
  \item merge - ``pushout property''/reconcile, symmetric (for distributed
    systems), (do we need associativity as well?)
  \end{itemize}
\item hopes/goals
  \begin{itemize}
  \item repos - modular/composable, somehow polymorphic
  \item patches - \emph{semantic} in some sense
  \item merge - easily definable [sic.], considers semantics of patches
  \end{itemize}
\end{enumerate}

\subsection{Results}
\begin{enumerate}
\item the idea
  \begin{itemize}
  \item index patches by endpoints
  \item define merges by Kraus \& Co's elimination rule
  \item ... profit?
  \end{itemize}
\item the result
  \begin{itemize}
  \item doesn't really work out (maybe some agda code here?)
  \item hard to say what the elimination principle actually says
  \item equivalences of equivalences is rough
  \end{itemize}
\end{enumerate}
This section describes our solution and why it doesn't work.

Basically: equivalences of equivalences is complicated, maybe hpt had a point
about reversibility