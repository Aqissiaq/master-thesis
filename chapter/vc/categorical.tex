\subsection{A Categorical Theory of Patches}

[THIS ISN'T REALLY RELEVANT TO ANYTHING ELSE, BUT IT'S A COOL IDEA]

A Categorical Theory of Patches~\cite{Categorical2013} defines a category of
files and patches, such that a merge is a pushout. To ensure a merge is always
possible they first construct the category $\mathcal{L}$ of files and patches,
and then its conservative cocompletion $\mathcal{P}$.

$\mathcal{P}$ contains all finite colimits -- and in particular all pushouts --
so the merge of a span is always defined. The paper's chief achievement is the
explicit construction of this category and these pushouts.

Interesting insights I'm not sure how to incorporate:
\begin{itemize}
\item the construction of $\mathcal{P}$ can be understood as the addition of \emph{partially} ordered
  files to $\mathcal{L}$.
\item ``flattening'' these partial orders leads to cyclic graphs. On editing
  text~\cite{editing2014} objects, but maybe not correctly
\item the poset structure of $\mathcal{L}$ and $\mathcal{P}$ is given explicitly
  by $\mathcal{G}$ and the nerve functor $N_{\_}$ (!!).
\end{itemize}

(maybe mention Pijul~\cite{Pijul} (if so, figure out the relationship to ~\cite{Categorical2013}))
(maybe some figures go here)