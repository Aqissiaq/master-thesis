\subsection{Homotopical Patch Theory}
Homotopical Patch Theory~\cite{Angiuli2016} gives a formulation of patch theory
in homotopy type theory. A patch theory is represented by a higher
inductive type, and its interpretation by a function out of this type.

By representing repository state as points and patches as paths in a higher
inductive type, the groupoid structure of the patch theory comes ``for free''.
Paths come with composition, and by the groupoid laws this composition is
associative, unital, and respects inverses. Additionally, functions (which are functors)
respect this structure so any interpretation must also validate the groupoid
laws.

Patch laws are represented by paths between paths (squares? disks?
2D-somethings). For example we may want the application of two independent
patches to commute -- this is done with a patch law.

While the HIT formulation gives a lot ``for free'', it also has some drawbacks.
In particular, the requirement that all patches have inverses causes some
problems. The workaround is to ``type'' patches with the history they are
applicable to. This allows Angiuli et al. to define a merge operation in terms
on only the ``forward'' patches, but leads to a fairly complex theory even for
relatively simple settings.

% cut this if we don't do more work on it
An interesting feature of Angiuli et al.'s patch theories is that the type of
repositories must be contractible. Since patches are represented by paths, any
point can be retracted along them. As such, all repositories are -- in a sense
-- ``the same'' and we need better notions of
``sub-homotopical''~\cite{Angiuli2016} computations to reason about their differences.