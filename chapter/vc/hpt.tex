\subsection{Homotopical Patch Theory}\label{subsec:hpt}

Homotopical Patch Theory~\cite{Angiuli2016} (HPT) gives a way to formulate patch
theories in homotopy type theory. The formulation makes intrinsic use of higher
inductive types (HITs) and a univalent universe to encode relationships between
patches and models of a theory.

HPT takes advantage of the inherent groupoid structure of types to encode the
groupoid structure of patch theories ``for free''. A patch theory is given as a
higher inductive type \texttt{R} with points representing patch contexts and
paths representing patches between them. Paths come with units (\texttt{refl}),
inverses (\texttt{sym} [OR MAYBE $^{-1}$]) and composition (\texttt{$\cdot$}),
and by the groupoid laws this composition is associative, unital, and respects
inverses.

Patch laws can be given by paths between paths (squares). For example we may
want the application of two independent patches $P$ and $Q$ to commute -- this is done with
a square whose left- and right-sides are $P \cdot Q$ and $Q \cdot P$.

\emph{Models} of a patch theory are given by a function \texttt{I : R
  $\rightarrow$ Type} from the HIT to the universe of types. Each repository
state (point) is mapped to a type, and each patch is mapped to a path in the
universe. By univalence such paths can be given by equivalences between types,
and each patch $P : x \equiv y$ gives rise to a function \texttt{interp P : I(x)
$\rightarrow$ I(y)} by \texttt{transport}. The functoriality of \texttt{I}
ensures that such a model validates all the patch laws of \texttt{R}.

Angiuli et al. present three example patch theories in order of increasing
complexity. Implementations in Cubical Agda are explored in more depth in
\autoref{ch/formalization}, but we give a brief overview here.

\subsubsection{An Elementary Patch Theory}

An elementary patch theory describes a VCS with one context \texttt{num}, and
one patch \texttt{add1 : num $\equiv$ num}. Being a HIT\footnote{This HIT with one point and
one non-identity loop may seem familiar. It is a renaming of the circle $S^1$!},
the theory automatically includes identity patches, composition and even inverses.

The intended interpretation of the theory is a repository consisting of a single
integer where applying \texttt{add1} adds 1 to it. However, there is nothing
stopping us from giving a different interpretation. [MAYBE DO THAT? LATER?]

This example illustrates two important points. Firstly, that paths can carry
computational content (in this case the successor function and its inverse)
revealed by mapping into the universe of types; and secondly that a
patch theory may permit multiple models.

\subsubsection{A Patch Theory with Laws}

In \textbf{a patch theory with laws} we see an example of patch laws -- squares
in the higher inductive type. Again the theory has one point and one constructor
path, but the path \texttt{s $\leftrightarrow$ t @ i} depends on two
strings \texttt{s,t} and an index \texttt{i}.

The theory also includes two patch laws. \texttt{indep} asserts that two patches
with different indexes commute, while \texttt{noop} asserts that a patch where
$s = t$ is the same as \texttt{refl} - doing nothing.

The intended model for this theory is a fixed-length vector of strings where the
patch \texttt{s $\leftrightarrow$ t @ i} swaps the strings \texttt{s} and
\texttt{t} at index \texttt{i}. In order to define this interpretation it is
also necessary to give squares showing that it respects the patch laws.

The use of patch laws is illustrated by a \textbf{patch optimizer}. This is a
function that takes a patch and produces a smaller patch with the same effect.
The key idea is to take advantage of \texttt{noop} to replace instances of
\texttt{s $\leftrightarrow$ t @ i} with \texttt{refl} when \texttt{s} and
\texttt{t} are equal.

Angiuli et al. give two ways to write the optimizer. In the \textbf{program then
  prove} approach they construct a function \texttt{opt1 : R $\rightarrow$ R}
and then prove \texttt{$\forall$ x $\rightarrow$ x $\equiv$ opt1 x}. This
requires set-truncation of \texttt{R}.

The \textbf{program and prove} approach avoids truncation by instead
constructing a function \texttt{opt : (x : R) $\rightarrow$ $\Sigma_\texttt{(y :
    R)}$ y $\equiv$ x} which produces both a new point \emph{and} and proof that
it is equal to the original. Since the resulting type is contractible, the
squares witnessing the patch laws become trivial.

In both cases the actual \emph{patch} optimizer is obtained by applying
\texttt{opt} along a path.

\subsubsection{A Patch Theory with Richer Contexts} 

The previous two examples show the utility of a patch theory as a HIT,
but they do not capture the importance of contexts. In both, there is only one
context and every patch is applicable to that context.

\textbf{A patch theory with richer contexts} has contexts \texttt{doc h} indexed by a
\emph{history} \texttt{h}, and the two kinds of patches \texttt{add s i} and
\texttt{rm i} are only applicable to appropriate histories. In particular, a
history has a ``length'' \texttt{n} and patches are only applicable when their
index \texttt{i} is smaller than \texttt{n}.

This patch theory also has patch laws describing the relationships between
adding and removing lines in different orders. We leave out the details of these
laws for now, but note that the histories must also respect patch laws.

The interpretation of \texttt{doc h} for a history of length \texttt{n} is a
single file containing \texttt{n} lines of text. What about patches? The
\texttt{rm} patch should be a path between files of length \texttt{n} and files
of length \texttt{n-1} but these types are not bijective. To solve this we
compute the unique result of applying a history and map \texttt{doc h} to the
singleton type of the result. Now a bijection can be obtained, since all
singleton types are in bijection with each other.

Angiuli et al. then illustrates an interesting use of models as functions by
defining two different models for this theory. The first model computes the
resulting file as a vector of strings, while the second instead produces a log
of all the patches that have been applied.

Merging in this richer theory is reduced to a function on histories from the
empty file. This ensures that only patches in the ``forward direction'' are
merged, and since histories respect patch laws so does their merge.

\subsubsection{Computational Content of Paths}

In all three examples, the models of patch theories make crucial use of
different paths. In an elementary patch theory \texttt{refl}, \texttt{add1} and
\texttt{add1 $\cdot$ add1} all have the same endpoints, but their effects as
patches are very different. Therefore making use of this formulation requires a
type theory where paths (and in particular paths between types provided by univalence) carry
computational meaning, such as Cubical Agda.

Additionally, the optimizer for a patch theory with laws maps into a
contractible type in a non-trivial way. This may seem pointless, since all the
elements in such a type are in a sense the same\footnote{In fact any
  function of the type \texttt{(x : R) $\rightarrow$ $\Sigma_\texttt{(y:R)}$ x
  $\equiv$ y} is homotopic to the ``identity function'' that maps x to (x ,
refl).~\cite{Angiuli2016}}, but nevertheless we expect it to compute the correct
patch. In practice we require some notion of ``sub-homotopical'' computation.

For these reasons it is worthwhile to implement the models cubically to gain a
better understanding both of homotopical patch theories and cubical
equality~\cite{vezzosi2021cubical, Angiuli2016}.

%% Below are some paragraphs from an earlier version
%% not sure if/when they will be useful


% While the HIT formulation gives a lot ``for free'', it also has some drawbacks.
% In particular, the requirement that all patches have inverses causes some
% problems. The workaround is to ``type'' patches with the history they are
% applicable to. This allows Angiuli et al. to define a merge operation in terms
% on only the ``forward'' patches, but leads to a fairly complex theory even for
% relatively simple settings.

% % cut this if we don't do more work on it
% % this also isn't quite true, and the sub-homotopical part refers to opt
% An interesting feature of Angiuli et al.'s patch theories is that the type of
% repositories must be contractible. Since patches are represented by paths, any
% point can be retracted along them. As such, all repositories are -- in a sense
% -- ``the same'' and we need better notions of
% ``sub-homotopical''~\cite{Angiuli2016} computations to reason about their differences.